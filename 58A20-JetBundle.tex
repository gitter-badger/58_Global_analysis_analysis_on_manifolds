\documentclass[12pt]{article}
\usepackage{pmmeta}
\pmcanonicalname{JetBundle}
\pmcreated{2013-03-22 15:28:42}
\pmmodified{2013-03-22 15:28:42}
\pmowner{rspuzio}{6075}
\pmmodifier{rspuzio}{6075}
\pmtitle{jet bundle}
\pmrecord{8}{37333}
\pmprivacy{1}
\pmauthor{rspuzio}{6075}
\pmtype{Definition}
\pmcomment{trigger rebuild}
\pmclassification{msc}{58A20}
\pmdefines{jet}

% this is the default PlanetMath preamble.  as your knowledge
% of TeX increases, you will probably want to edit this, but
% it should be fine as is for beginners.

% almost certainly you want these
\usepackage{amssymb}
\usepackage{amsmath}
\usepackage{amsfonts}

% used for TeXing text within eps files
%\usepackage{psfrag}
% need this for including graphics (\includegraphics)
%\usepackage{graphicx}
% for neatly defining theorems and propositions
%\usepackage{amsthm}
% making logically defined graphics
%%\usepackage{xypic}

% there are many more packages, add them here as you need them

% define commands here
\def\J{\mathrm{J}}
\def\T{\mathrm{T}}
\begin{document}
\PMlinkescapeword{between}
\PMlinkescapeword{structure}

\subsection{Introduction}
Let $p\colon X\to B$ be a surjective submersion of $\mathcal{C}^r$
differential manifolds, where
$r\in\{0,1,\ldots\}\cup\{\infty,\omega\}$ ($\mathcal{C}^\omega$ means
real analytic).  For all integers $k\ge 0$ with $k\le r$, we will
define a fibre bundle $\J^k_B X$ over $X$, called the $k$-th \emph{jet
bundle} of $X$ over $B$.  The fibre of this bundle above a point $x\in
X$ can be interpreted as the set of equivalence classes of local
sections of $p$ \PMlinkescapetext{passing through} $x$, where two
sections are considered
equivalent if their first $k$ derivatives at $x$ are equal.  The
equivalence class of a section is then the \emph{jet} of that section;
it indicates the direction of the section locally at $x$.  This
concept has much in common with that of the germ of a smooth function
on a manifold: it \PMlinkescapetext{contains} not only the value of a
function at a point, but also some \PMlinkescapetext{information}
about the behaviour of the function near that point.

\subsection{Construction}
We will now define each jet bundle of $X$ over $B$ as a set with a
projection map to $X$, and we describe the concept of
\emph{prolongation} of sections.  After that, we give a slightly
different construction allowing us to put a manifold structure on each
of the jet bundles.

For every open subset of $B$, we denote by $\Gamma(U,X)$ the set of
sections of $p$ over $U$, i.e.~the set of ${\cal C}^r$ functions
$s\colon U\to p^{-1}U$ such that $p\vert_{p^{-1}U}\circ s=\mathop{\rm
id}_U$.  Every point of $B$ has an open subset $U$ such that there
exists at least one section of $p$ over $U$, due to the assumption
that $p$ is a surjective submersion.  For all $x\in X$, we define the
fibre of $\J^k_BX$ above $x$ by
$$
\J^k_BX(x)=\{(U,s)\colon U\subset B\hbox{ open},
x\in U,s\in\Gamma(U,X),s(p(x))=x\}/\mathord\sim,
$$
where the equivalence relation $\sim$ is defined by $s\sim
s'\Leftrightarrow s$ and $s'$ induce the same map between the fibres
at $p(x)$ and $x$ of the $k$-th iterated tangent bundles of $B$ and
$X$, respectively.  (Note that the fibres in the $k$-th iteration are
the same if and only if the induced maps are already the same in the
$(k-1)$-st iteration).  We will denote the equivalence class of a pair
$(U,s)$ by $[U,s]$.  As a set, $\J^k_BX$ is defined as the disjoint
union of the sets $\J^k_BX(x)$ with $x\in X$.  Write $\pi\colon
\J^k_BX\to X$ for the `obvious' projection map, defined by
$$
\pi([U,s])=x\hbox{ for }[U,s]\in\J^k_BX(x).
$$
Notice that $\J^0_BX$ is just $X$ itself.

Suppose we have some section $s$ of $X$ over an open subset $U$ of
$B$.  By sending every point $y\in U$ to the equivalence class
$[U,s]\in\J^k_BX(s(y))$ we obtain a section of $\pi\circ p\colon
\J^k_BX\to B$ over $U$ for each $k\ge 0$, called the $k$-th
\emph{prolongation} of $s$.  Composing this section on the left with
$\pi$ gives back the original section $s$ of $X$.

\subsection{The bundle structure}
Instead of defining all the jet bundles at once, we may choose to
define only the first jet bundle in the way described above.  After
equipping the first jet bundle with the structure of a differential
manifold, which we will do below, we can then inductively define
$\J^{k+1}_BX$ as the first jet bundle of $\J^k_BX$ over $B$ for $k\ge
1$.  This is useful because the manifold structure only needs to be
defined for $\J^1_BX$.  %It is easy to check that both constructions give isomorphic results.

We make $\J^1_BX$ into an affine bundle over $X$, locally trivial of
rank $\dim B(\dim X-\dim B)$, in the following way.  We cover $X$ with
charts $(W,\phi,W')$, where $\phi$ is a diffeomorphism between open
subsets $W\subset X$ and $W'\subset\mathbb{R}^n$.  Without loss of
generality, we assume that $p(W)$ is contained in the domain $V$ of a
chart $(V,\psi,V')$ on $B$, with $V'\subset\mathbb{R}^m$.  Here $n$
and $m$ are the local dimensions of $X$ and $B$, respectively.

For all $x\in W$ and all $[U,s]\in\J^1_BX(x)$, we have the tangent map
$\T s(p(x))$, which is a linear map from $\T V(p(x))$ to $\T W(x)$.
These tangent spaces are isomorphic to $\mathbb{R}^m$ and
$\mathbb{R}^n$ via the chosen charts, so that $\T s(p(x))$ acts as a
matrix $M_x([U,s])\in\mathbb{R}^{n\times m}$:
$$
\xymatrix{
\T W(x) \ar[r]^\simeq & \mathbb{R}^n \\
\T V(p(x)) \ar[u]^{\T s(p(x))} \ar[r]^\simeq &
\mathbb{R}^m \ar[u]_{M_x([U,s])}
}
$$
The definition of the equivalence relation $\sim$ on $\J^1_BX(x)$ means
that the association $[U,s]\mapsto M_x([U,s])$ is well-defined and
injective for each $x\in W$.  The image of $M_x$ consists of the
matrices with the property that multiplying them on the left with the
$m\times n$ matrix corresponding to the tangent map $\T p(x)$ gives
the $m\times m$ identity matrix.  These matrices form a
$m(n-m)$-dimensional linear subspace $L_x\subset\mathbb{R}^{n\times m}$,
and $L=\bigcup_{x\in W}\{x\}\times L_x$ is a submanifold of
$W\times\mathbb{R}^{n\times m}$.

We \PMlinkescapetext{fix} both the differentiable structure of
$\pi^{-1}U$ and a local trivialisation of $\J^1_BX$ as a vector bundle
by requiring that
\begin{eqnarray*}
\pi^{-1}W&\to&L \\
{}[U,s] &\mapsto&\left(\pi([U,s]),M_{\pi([U,s])}([U,s])\right)
\end{eqnarray*}
be a diffeomorphism and an $\mathbb{R}$-linear map.  Since the effect
of a change of charts on $W$ or $V$ is multiplying each $M_x([U,s])$ by
matrices depending differentiably on $x\in W$ (namely, the derivatives
of the glueing maps), this gives a well-defined vector bundle
structure on all of $\J^1_BX$.

Iterating the above construction by defining $\J^{k+1}_BX$ as the first
jet bundle of $\J^k_BX$ over $B$, each jet bundle $\J^{k+1}_BX$ becomes
a vector bundle over $\J^k_BX$ and a fibre bundle over $X$.  Normally,
only $\J^1_BX$ is a vector bundle over $X$.
%%%%%
%%%%%
\end{document}
