\documentclass[12pt]{article}
\usepackage{pmmeta}
\pmcanonicalname{DifferentialForm}
\pmcreated{2013-03-22 12:44:46}
\pmmodified{2013-03-22 12:44:46}
\pmowner{rmilson}{146}
\pmmodifier{rmilson}{146}
\pmtitle{differential form}
\pmrecord{28}{33050}
\pmprivacy{1}
\pmauthor{rmilson}{146}
\pmtype{Definition}
\pmcomment{trigger rebuild}
\pmclassification{msc}{58A10}
\pmdefines{exterior derivative}
\pmdefines{1-form}
\pmdefines{exterior product}
\pmdefines{wedge product}
\pmdefines{interior product}
\pmdefines{tensorial}

\usepackage{amsmath}
\usepackage{amsfonts}
\usepackage{amssymb}

\newcommand{\Rset}{\mathbb{R}}
\newcommand{\Hom}{\operatorname{Hom}}
\begin{document}
\PMlinkescapeword{exterior}
\PMlinkescapeword{interior}
\section{Notation and Preliminaries.}
Let $M$ be an $n$-dimensional differential manifold.  Let $TM$ denote
the manifold's tangent bundle, $C^\infty(M)$ the algebra of smooth
functions, and $ V(M)$ the Lie algebra of smooth vector fields.  The
directional derivative  makes $C^\infty(M)$ into a $V(M)$
module.  Using local coordinates, the directional derivative operation
can be expressed as
\[ v(f) = v^i \partial_i f,\quad v\in V(M),\; f\in C^\infty(M).\]


\section{Definitions.}
\paragraph{Differential forms.} 
Let $A$ be a $C^\infty(M)$ module. An $ \mathbb{R}$-linear mapping $ \alpha: V(M)\to A$ is said to be \emph{tensorial} if it is a
$C^\infty(M)$-homomorphism, in other words, if it satisfies
\[  \alpha(f v) = f \alpha(v)  \]
for all
for all vector fields $v\in V(M)$ and functions $f\in C^\infty(M)$.
More generally, a multilinear map $\alpha: V(M)\times\dots\times
V(M)\to A$ is called tensorial if it satisfies
\[
  \alpha(f u,\dots,v)  =   \cdots = \alpha(u,\dots, fv) = f \alpha(u,\dots, v)
\]
for all vector fields $u,\dots,v$ and all functions $f\in C^\infty(M)$.



We now define a differential 1-form to be a tensorial linear mapping
from $V(M)$ to $C^\infty(M)$.  More generally, for $k=0,1,2,\ldots, $ we
define a differential $k$-form to be a tensorial multilinear,
antisymmetric, mapping from $V(M)\times \cdots \times V(M)$ ($k$
times) to $C^\infty(M)$.  Using slightly fancier language, the above amounts
to saying that a $1$-form is a section of the cotangent bundle $T^*M =
\Hom(TM,\Rset)$, while a differential $k$-form as a section of
$\Hom(\Lambda^k TM,\Rset)$. 

Henceforth, we let $\Omega^k(M)$ denote the $C^\infty(M)$-module of
differential $k$-forms. In particular, a differential $0$-form is the
same thing as a function.  Since the tangent spaces of $M$ are
$n$-dimensional vector spaces, we also have $\Omega^k(M)=0$ for $k>n$.
We let
\[\Omega(M) = \bigoplus_{k=0}^n \Omega^k(M)\]
denote the vector space of all differential forms.  There is a natural
operator, called the exterior product, that endows $\Omega(M)$ with
the structure of a graded algebra.  We describe this operation below.

  
\paragraph{Exterior and Interior Product.}
Let $v\in V(M)$ be a vector field and $\alpha\in \Omega^k(M)$ a
differential form.  We define $\iota_v(\omega)$, the interior product
of $v$ and $\alpha$, to be the differential $k-1$ form given by
\[ \iota_v(\alpha)(u_1,\dots,u_{k-1}) =
\alpha(v,v_1,\dots,v_{k-1}),\quad v_1,\dots,v_{k-1}\in V(M).\] The
interior product of a vector field with a $0$-form is defined to be
zero.  

Let $\alpha\in \Omega^k(M)$ and $\beta\in\Omega^\ell(M)$ be differential
forms.  We define the exterior, or wedge product
$\alpha\wedge\beta\in\Omega^{k+\ell}(M)$ to be the 
unique differential form such that
\[ \iota_v(\alpha\wedge\beta) = \iota_v(\alpha)\wedge \beta + (-1)^k
\alpha \wedge \iota_v(\beta)\]
for all vector fields $v\in V(M)$.  Equivalently, we could have defined
\[ (\alpha\wedge\beta)(v_1,\dots, v_{k+\ell}) =
\sum_{\pi}\operatorname{sgn}(\pi)
\alpha(v_{\pi_1},\dots,v_{\pi_k})
\beta(v_{\pi_{k+1}},\dots,v_{\pi_{k+\ell}}),\] where the sum is taken
over all permutations $\pi$ of $\{1,2,\dots, k+\ell\}$ such that $\pi_1<
\pi_2 < \cdots \pi_k$ and $\pi_{k+1} < \cdots < \pi_{k+\ell}$, and where
$\operatorname{sgn} \pi=\pm 1$ according to whether $\pi$ is an even
or odd permutation.



\paragraph{Exterior derivative.} The exterior derivative is a
first-order differential operator $d:\Omega^*(M)\rightarrow
\Omega^*(M)$, that can be defined as the unique linear mapping 
satisfying
\begin{align*}
  d(d \alpha)&=0, \qquad \alpha\in\Omega^k(M);\\
  \iota_V(df)&=v(f),\qquad  v\in V(M),\;f\in C^\infty(M);\\
  d(\alpha \wedge \beta) &= d(\alpha)\wedge \beta + (-1)^k
  \alpha\wedge d(\beta), \qquad
  \alpha\in\Omega^k(M),\;\beta\in \Omega^\ell(M).
\end{align*}


\section{Local coordinates.}
Let $(x^1,\ldots,x^n)$ be a system of local coordinates on $M$, and
let $\partial_1,\dots,\partial_n$ denote the corresponding frame of
coordinate vector fields. In other words, 
\[ \partial_i(x^j) = \delta_i{}^j,\] where the right hand side is the
usual Kronecker delta symbol. By the definition of the
exterior derivative, 
\[\iota_{\partial_i} (dx^j) = \delta_i{}^j;\]
In other words, the 1-forms $dx^1,\dots,dx^n$ form the dual coframe.

Locally, the $\partial_i$ freely generate $V(M)$, meaning
that every vector field $v\in V(M)$ has the form
\[ v= v^i \partial_i, \]
where the coordinate components $v^i$ are uniquely determined as
\[ v^i=v(x^i). \]
Similarly, locally the $dx^i$ freely generate $\Omega^1(M)$.  This
means that 
every
one-form $\alpha\in\Omega^1(M)$ takes the form
\[\alpha=\alpha_i dx^i,\]
where
\[\alpha_i=\iota_{\partial_i} (\alpha).\]
More generally, locally $\Omega^k(M)$ is a freely generated by the
differential $k$-forms
\[dx^{i_1}\wedge\cdots\wedge dx^{i_k},\qquad 1\leq
i_1<i_2<\cdots<i_k\leq n.\] Thus, a differential form $\alpha\in
\Omega^k(M)$ is given by
\begin{align}
  \label{eq:dformcomponents}
  \alpha &= \!\!\!\sum_{i_1< \ldots< i_k} \!\!\!\alpha_{i_1\ldots
    i_k}\, d x^{i_1} \wedge\ldots\wedge d x^ {i_k},\\ \nonumber
  &= \frac{1}{k!} \; \alpha_{i_1\ldots
    i_k}\, d x^{i_1} \wedge\ldots\wedge d x^ {i_k},
\end{align}
where
\[ \alpha_{i_1\dots i_k} = \alpha(\partial_{i_1},\dots,
\partial_{i_k}).\] Consequently, for vector fields $u,v,\dots, w\in
V(M)$, we have
\[ \alpha(u,v,\dots,w)  = \alpha_{i_1i_2\dots i_k} u^{i_1} v^{i_2}\cdots w^{i_k}.\]

In terms of local coordinates and the skew-symmetrization index
notation, the interior and exterior product, and the exterior
derivative take the following expressions:
\begin{align}
  (\iota_v (\alpha))_{i_1\dots i_k} &= v^j
  \alpha_{j i_1\dots i_k},\quad v\in V(M),\; \alpha\in\Omega^{k+1}(M);\\
  \label{eq:dfextprod}
  (\alpha\wedge\beta)_{i_1\dots i_{k+\ell}} &= \binom{k+\ell}{k}\,
  \alpha_{[i_1\dots i_k} \beta_{i_{k+1}\dots i_{k+\ell}]},\quad
  \alpha\in\Omega^k(M),\; \beta\in\Omega^\ell(M);\\
  \label{eq:dfextder}
  (d\alpha)_{i_0i_1\dots i_k} &= (k+1) \,\partial_{[i_0} \alpha_{i_1\dots
    i_k]},\quad \alpha\in\Omega^k(M).
\end{align}
Note that some authors prefer a different definition of the components
of a differential.  According to this alternate convention, a factor
of $k!$ placed before the summation sign in
\eqref{eq:dformcomponents}, and the leading factors are removed from
\eqref{eq:dfextprod} and \eqref{eq:dfextder}.
%%%%%
%%%%%
\end{document}
