\documentclass[12pt]{article}
\usepackage{pmmeta}
\pmcanonicalname{Poincare1form}
\pmcreated{2013-03-22 14:45:44}
\pmmodified{2013-03-22 14:45:44}
\pmowner{matte}{1858}
\pmmodifier{matte}{1858}
\pmtitle{Poincar\'e $1$-form}
\pmrecord{7}{36405}
\pmprivacy{1}
\pmauthor{matte}{1858}
\pmtype{Definition}
\pmcomment{trigger rebuild}
\pmclassification{msc}{58A32}
\pmsynonym{Liouville one-form}{Poincare1form}

\endmetadata

% this is the default PlanetMath preamble.  as your knowledge
% of TeX increases, you will probably want to edit this, but
% it should be fine as is for beginners.

% almost certainly you want these
\usepackage{amssymb}
\usepackage{amsmath}
\usepackage{amsfonts}
\usepackage{amsthm}

\usepackage{mathrsfs}

% used for TeXing text within eps files
%\usepackage{psfrag}
% need this for including graphics (\includegraphics)
%\usepackage{graphicx}
% for neatly defining theorems and propositions
%
% making logically defined graphics
%%%\usepackage{xypic}

% there are many more packages, add them here as you need them

% define commands here

\newcommand{\sR}[0]{\mathbb{R}}
\newcommand{\sC}[0]{\mathbb{C}}
\newcommand{\sN}[0]{\mathbb{N}}
\newcommand{\sZ}[0]{\mathbb{Z}}

 \usepackage{bbm}
 \newcommand{\Z}{\mathbbmss{Z}}
 \newcommand{\C}{\mathbbmss{C}}
 \newcommand{\R}{\mathbbmss{R}}
 \newcommand{\Q}{\mathbbmss{Q}}



\newcommand*{\norm}[1]{\lVert #1 \rVert}
\newcommand*{\abs}[1]{| #1 |}



\newtheorem{thm}{Theorem}
\newtheorem{defn}{Definition}
\newtheorem{prop}{Proposition}
\newtheorem{lemma}{Lemma}
\newtheorem{cor}{Corollary}
\begin{document}
\begin{defn} Suppose $M$ is a manifold, and $T^\ast M$ is its cotangent bundle.
Then the \PMlinkescapetext{\emph{Poincar\'e $1$-form}}, 
$\alpha \in \Omega^1(T^\ast M)$, is locally
defined as
$$ 
  \alpha = \sum_{i=1}^n y_i dx^i   
$$
where $x^i, y_i$ are canonical local coordinates for $T^\ast M$.
\end{defn}

Let us show that the Poincar\'e $1$-form is globally defined. That is, 
$\alpha$ has the same expression in all local coordinates. Suppose $x^i, \tilde{x}^i$ are overlapping coordinates for $M$. Then we have 
overlapping local coordinates $(x^i, y_i)$, $(\tilde{x}^i, \tilde{y}_i)$
for $T^\ast M$ with the transformation rule
$$
   \tilde{y}_i = \frac{\partial \tilde{x}^j}{\partial x^i} y_j.
$$
Hence
\begin{eqnarray*}
\sum_{i=1}^n \tilde{y}_i d\tilde{x}^i &=& \sum_{i=1}^n \tilde{y}_i \frac{\partial \tilde{x}^i}{\partial x^k} dx^k \\
&=& \sum_{i=1}^n \frac{\partial \tilde{x}^j}{\partial x^i} y_j \frac{\partial \tilde{x}^i}{\partial x^k} dx^k  \\
&=& \sum_{k=1}^n y_k dx^k.
\end{eqnarray*}

\subsubsection*{Properties}
\begin{enumerate}
\item The Poincar\'e $1$-form play a crucial role in symplectic geometry. 
The form $d\alpha$ is the canonical symplectic form for $T^\ast M$. 
\item Suppose $\pi\colon T^\ast M\to M$ is the canonical projection. 
Then
$$
  \alpha(w) = \xi( (D\pi)(w) ),\quad w\in T_\xi(T^\ast M),
$$
which is an alternative definition of $\alpha$ without local coordinates.
\item The restriction of this form to the unit cotangent bundle, is a 
contact form.
\end{enumerate}
%%%%%
%%%%%
\end{document}
