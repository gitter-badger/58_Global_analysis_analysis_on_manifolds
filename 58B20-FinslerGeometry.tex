\documentclass[12pt]{article}
\usepackage{pmmeta}
\pmcanonicalname{FinslerGeometry}
\pmcreated{2013-03-22 15:03:37}
\pmmodified{2013-03-22 15:03:37}
\pmowner{paolini}{1187}
\pmmodifier{paolini}{1187}
\pmtitle{Finsler geometry}
\pmrecord{8}{36780}
\pmprivacy{1}
\pmauthor{paolini}{1187}
\pmtype{Definition}
\pmcomment{trigger rebuild}
\pmclassification{msc}{58B20}
\pmclassification{msc}{53B40}
\pmclassification{msc}{53C60}
\pmrelated{WulffTheorem}

\endmetadata

% this is the default PlanetMath preamble.  as your knowledge
% of TeX increases, you will probably want to edit this, but
% it should be fine as is for beginners.

% almost certainly you want these
\usepackage{amssymb}
\usepackage{amsmath}
\usepackage{amsfonts}

% used for TeXing text within eps files
%\usepackage{psfrag}
% need this for including graphics (\includegraphics)
%\usepackage{graphicx}
% for neatly defining theorems and propositions
%\usepackage{amsthm}
% making logically defined graphics
%%%\usepackage{xypic}

% there are many more packages, add them here as you need them

% define commands here
\begin{document}
Let $\mathcal M$ be an
 $n$-dimensional differential manifold and let $\phi\colon T\mathcal M \to \mathbb R$ be a function $\phi(x,\xi)$ defined for $x\in \mathcal M$ and $\xi \in T_x \mathcal M$ such that $\phi(x,\cdot)$ is a possibly non symmetric norm on $T_x\mathcal M$. The couple $(\mathcal M, \phi)$ is called a Finsler space.

Let us define the $\phi$-length of curves in $\mathcal M$. If $\gamma\colon [a,b]\to\mathcal M$ is a differentiable curve we define
\[
  \ell_\phi(\gamma) := \int_a^b \phi(\gamma'(t))\, dt.
\]

So a natural geodesic distance can be defined on $\mathcal M$ which turns the Finsler space into a quasi-metric space (if $\mathcal M$ is connected):
\[
  d_\phi(x,y):= \inf\{\ell_\phi(\gamma)\colon \text{$\gamma$ is a differentiable curve $\gamma\colon[a,b]\to\mathcal M$ such that $\gamma(a)=x$ and $\gamma(b)=y$}\}.
\]

Notice that every Riemann manifold $(\mathcal M,g)$ is also a Finsler space, 
the norm $\phi(x,\cdot)$ being the norm induced by the scalar product $g(x)$.

A finite dimensional Banach space is another simple example of Finsler space, where $\phi(x,\xi):=\Vert \xi \Vert$. Wulff Theorem is one of the most important theorems in this ambient space.
%%%%%
%%%%%
\end{document}
