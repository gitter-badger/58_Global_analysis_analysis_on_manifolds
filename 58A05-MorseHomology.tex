\documentclass[12pt]{article}
\usepackage{pmmeta}
\pmcanonicalname{MorseHomology}
\pmcreated{2013-03-22 15:21:09}
\pmmodified{2013-03-22 15:21:09}
\pmowner{PrimeFan}{13766}
\pmmodifier{PrimeFan}{13766}
\pmtitle{Morse homology}
\pmrecord{17}{37174}
\pmprivacy{1}
\pmauthor{PrimeFan}{13766}
\pmtype{Definition}
\pmcomment{trigger rebuild}
\pmclassification{msc}{58A05}

% this is the default PlanetMath preamble.  as your knowledge
% of TeX increases, you will probably want to edit this, but
% it should be fine as is for beginners.

% almost certainly you want these
\usepackage{amssymb}
\usepackage{amsmath}
\usepackage{amsfonts}

% used for TeXing text within eps files
%\usepackage{psfrag}
% need this for including graphics (\includegraphics)
%\usepackage{graphicx}
% for neatly defining theorems and propositions
%\usepackage{amsthm}
% making logically defined graphics
%%%\usepackage{xypic}

% there are many more packages, add them here as you need them

% define commands here
\DeclareMathOperator{\Id}{Id}
\DeclareMathOperator{\Crit}{Crit}
\begin{document}
{\em Morse homology} is a tool developed by Thom, Smale, and Milnor for homology theory.

Take $M$ to be a smooth compact manifold. Throughout we assume that $f$ is a suitable Morse function, that is, all critical points of $f$ are nondegenerate.  We must first make some definitions before defining the Morse homology.
Choose a Riemannian metric on $M$ so that the notion of a gradient vector field makes sense.
The map $\phi\colon\mathbb{R} \times M \rightarrow M$ such that 
\[
\frac{d}{dt}\phi(t,x) = -\nabla f(\phi(t,x)),
\]
with $\phi(0,x) = \Id$, is called the negative gradient flow of $f$. Let $p$ be a critical point of $f$, and define 
\[
W_p^s :=  \{ x \in M | \lim_{t\rightarrow \infty} \phi(t,x) = p \} \text{\ and \ } W_p^u := \{ x\in M | \lim_{t \rightarrow -\infty} \phi(t,x) = p \}
\]
to be the stable and unstable manifolds respectively.
Thom realized that one could decompose $M$ into its unstable manifolds and arrive at something that is homologically equivalent to its handle decomposition, but this decomposition was not a CW complex, hence it was hard to say anything about the homotopy type of $M$. But Smale realized that if we impose more conditions on the metric itself, then we can make this into a CW complex.

The pair $(f,g)$, where $f$ is a Morse function and $g$ is the Riemannian metric, is called Morse-Smale pair, if for every pair $p$, $q$ of critical points of $f$, $W_p^u$ is transverse to $W_q^s$. This is known as the Morse-Smale condition.
This condition actually holds for a generic Riemannian metric on M. With this restriction, this makes Thom's decomposition into a CW complex.

We can define a complex called the Morse complex as follows: 

Let $\Crit_k (f)$ be the set of critical points of $f$ of index $k$. We define the chain group , $C_k(f)$ to be the formal linear combination with integer coefficients of elements of $\Crit_k (f)$. We must also keep track of the signs of the flow lines. (However, it is true if you count mod 2, the Morse complex computes homology with coefficients in $Z \over 2$.)
To make this a chain complex we must define the differential map. 
The map $\delta _k : C_k \rightarrow C_{k-1}$ applied to a critical point $p$ is a formal sum of critical points with $q$ given by this number. It is possible to prove that $\delta^2 = 0$ , making this into a chain complex.

The homology of this complex is called the Morse homology. It can be shown to be isomorphic to the singular homology of $M$. 

Note: There is another way of realizing the Morse homology using Hodge theory, an idea pioneered by Edward Witten. His idea is essentially to conjugate the $d$ operator by $e^{sf}$ and it can be shown that this conjugation again leads to another isomorphism between the set of harmonic forms and the De Rham cohomology. This parameter $s$ is like a curve of chain complexes and Witten claimed that if $s$ is large enough, then we can obtain a space whose dimension is the number of critical points of a given index and the boundary operator induced on $d$ is the number of critical paths between critical points, as before. Witten did not prove this idea rigorously, but it was done later by Helffer and Sjostrand.
%%%%%
%%%%%
\end{document}
