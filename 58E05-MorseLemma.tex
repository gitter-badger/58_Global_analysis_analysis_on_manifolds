\documentclass[12pt]{article}
\usepackage{pmmeta}
\pmcanonicalname{MorseLemma}
\pmcreated{2013-03-22 13:53:12}
\pmmodified{2013-03-22 13:53:12}
\pmowner{matte}{1858}
\pmmodifier{matte}{1858}
\pmtitle{Morse lemma}
\pmrecord{18}{34631}
\pmprivacy{1}
\pmauthor{matte}{1858}
\pmtype{Theorem}
\pmcomment{trigger rebuild}
\pmclassification{msc}{58E05}
%\pmkeywords{morse theory}
\pmdefines{non degenerate critical point}
\pmdefines{index of a bilinear map}

\endmetadata

% this is the default PlanetMath preamble.  as your knowledge
% of TeX increases, you will probably want to edit this, but
% it should be fine as is for beginners.

% almost certainly you want these
\usepackage{amssymb}
\usepackage{amsmath}
\usepackage{amsfonts}

% used for TeXing text within eps files
%\usepackage{psfrag}
% need this for including graphics (\includegraphics)
%\usepackage{graphicx}
% for neatly defining theorems and propositions
%\usepackage{amsthm}
% making logically defined graphics
%%%\usepackage{xypic}

% there are many more packages, add them here as you need them

% define commands here
\newtheorem{thm}{Theorem}
\newtheorem{prop}{Proposition}

\newcommand{\C}{\mathbb{C}}
\newcommand{\ddd}{\mathrm{d}}
\newcommand{\Hom}[2]{\mathrm{Hom}(#1,#2)}
\newcommand{\id}{\mathrm{id}}
\renewcommand{\ker}{\mathrm{ker}\,}
\newcommand{\N}{\mathbb{N}}
\newcommand{\Q}{\mathbb{Q}}
\newcommand{\R}{\mathbb{R}}
\newcommand{\Z}{\mathbb{Z}}

\begin{document}
\newcommand{\criti}{\mathrm{Crit}}
\newcommand{\inde}{\mathrm{Index}}
\newcommand{\deriv}[1]{\frac{\partial }{\partial #1}}
\newcommand{\dderiv}[3]{\frac{\partial^2 #1}{\partial #2\,\partial #3}}
\newcommand{\derivv}[2]{\left.\deriv{#1}\right|_{#2}}



Let    $M$ be a smooth $n$-dimensional manifold, and $f:M\rightarrow \mathbb \R$ a
smooth map. We denote by $\criti(f)$ the set of critical points of
$f$, i.e. $$\criti(f)=\{p\in M\,|\, (f_*)_p=0\}$$

For each $p\in \criti(f)$ we denote by $f_{**}:T_pM\times
T_pM\rightarrow\R$ (or $(f_{**})_p$ if $p$ need to be specified) the
bilinear map $$f_{**}(v,w)=v(\tilde w (f))=w(\tilde v (f)),\ \
\forall v,w\in T_pM,$$ where $\tilde v,\tilde w\in\mathcal T(M)$ are
smooth vector fields such that $\tilde v_p=v$ and $\tilde w_p=w$.
This is a good definition. In fact $p\in\criti(f)$ implies
$$v(\tilde w (f))-w(\tilde v (f))=(\tilde v (f),\tilde
w (f))_p=f_{*}(\tilde v,\tilde w)_p=0.$$ In smooth local coordinates
$x^1,...,x^n$ on a neighborhood $U$ of $p$ we have
$$f_{**}\left( \derivv{x^i}{p}, \derivv{x^j}{p}
\right)=\dderiv{f}{x^i}{x^j}(p).$$

A critical point $p\in\criti(f)$ is called \emph{non degenerate}
when the matrix $$\left( \dderiv{f}{x^i}{x^j}(p)
\right)_{i,j\in\{1,...,n\}}$$ is non singular. We can equivalently
express this condition without the use of local coordinates saying
that $p\in\criti(f)$ is non degenerate when for each $v\in T_p
M\setminus\{0\}$ the linear functional $f_{**}(v,\cdot)\in\Hom{T_pM}{\R}$
is not zero, i.e. there exists $w$ such that $f_{**}(v,w)\neq 0$.

We recall that the \emph{index} of a bilinear functional $H:V\times
V\rightarrow\R$ is the dimension $\inde(H)$ of a maximal linear
subspace $W\subseteq V$ such that $H$ is negative definite on
$W\times W$.

\begin{thm}[Morse lemma]
Let $f:M\rightarrow\R$ be a smooth map. For each non degenerate $p\in\criti(f)$ there exists a neighborhood $U$ of $p$ and smooth coordinates $x=(x^1,...,x^n)$ on $U$ such that $x(p)=0$ and
$$f|_U=f(p)-(x^1)^2-...-(x^\lambda)^2+(x^{\lambda+1})^2+...+(x^n)^2,$$
where $\lambda=\inde((f_{**})_p)$.
\end{thm}
%%%%%
%%%%%
\end{document}
