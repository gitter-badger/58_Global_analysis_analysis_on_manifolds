\documentclass[12pt]{article}
\usepackage{pmmeta}
\pmcanonicalname{RectifiableCurrent}
\pmcreated{2013-03-22 14:28:34}
\pmmodified{2013-03-22 14:28:34}
\pmowner{paolini}{1187}
\pmmodifier{paolini}{1187}
\pmtitle{rectifiable current}
\pmrecord{4}{36001}
\pmprivacy{1}
\pmauthor{paolini}{1187}
\pmtype{Definition}
\pmcomment{trigger rebuild}
\pmclassification{msc}{58A25}
\pmdefines{integral current}
\pmdefines{integral flat norm}
\pmdefines{integral flat chains}

% this is the default PlanetMath preamble.  as your knowledge
% of TeX increases, you will probably want to edit this, but
% it should be fine as is for beginners.

% almost certainly you want these
\usepackage{amssymb}
\usepackage{amsmath}
\usepackage{amsfonts}

% used for TeXing text within eps files
%\usepackage{psfrag}
% need this for including graphics (\includegraphics)
%\usepackage{graphicx}
% for neatly defining theorems and propositions
%\usepackage{amsthm}
% making logically defined graphics
%%%\usepackage{xypic}

% there are many more packages, add them here as you need them

% define commands here
\begin{document}
An $m$-dimensional \emph{rectifiable current} is a current $T$ whose action against a $m$-form $\omega$ can be written as
\[
  T(\omega) = \int_S \theta(x)\langle \xi(x),\omega(x)\rangle d\mathcal H^m(x)
\]
where $S$ is an $m$-dimensional bounded rectifiable set, $\xi$ is an orientation of $S$ i.e.\ $\xi(x)$ is a unit $m$-vector representing the approximate tangent plane of $S$ at $x$ for $\mathcal H^m$-a.e.\ $x\in S$ and, finally, $\theta(x)$ is an integer valued measurable function defined a.e.\ on $S$ (called \emph{multiplicity}).
The space of $m$-dimensional rectifiable currents is denoted by $\mathcal R_m$.

An $m$-dimensional rectifiable current $T$ such that the boundary $\partial T$ is itself an $(m-1)$-dimensional rectifiable current, is called \emph{integral current}. The space of integral currents is denoted by $\mathbf I_m$.
We point out that the word ``integral'' refers to the fact that the multiplicity $\theta$ is integer valued.

Also notice that rectifiable and integral currents are not vector subspaces of the space of currents. In fact while the sum of two rectifiable currents is again a rectifiable current, the multiplication by a real number gives a rectifiable current only if the number is an integer.

The compactness theorem makes the space of integral currents a good space where geometric problems can be ambiented.

On rectifiable currents one can define an \emph{integral flat norm}
\[
  \mathcal F(T) := \inf \{\mathbf M(A) + \mathbf M(B)\colon
   T=A+\partial B,\quad A\in \mathcal R_m,\ B\in\mathcal R_{m+1}\}.
\]

The closure of the space $\mathcal R_m$ under the integral flat norm is called the space of \emph{integral flat chains} and is denoted by $\mathcal F_m$.

As a consequence of the closure theorem, one finds that $\mathcal R_m = \{T\in \mathcal F_m\colon \mathbf M(T)<\infty\}$ where $\mathbf M$ is the mass norm of a current.
%%%%%
%%%%%
\end{document}
