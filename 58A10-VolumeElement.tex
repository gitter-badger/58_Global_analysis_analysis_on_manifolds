\documentclass[12pt]{article}
\usepackage{pmmeta}
\pmcanonicalname{VolumeElement}
\pmcreated{2013-03-22 17:40:58}
\pmmodified{2013-03-22 17:40:58}
\pmowner{jirka}{4157}
\pmmodifier{jirka}{4157}
\pmtitle{volume element}
\pmrecord{5}{40122}
\pmprivacy{1}
\pmauthor{jirka}{4157}
\pmtype{Definition}
\pmcomment{trigger rebuild}
\pmclassification{msc}{58A10}
\pmclassification{msc}{53-00}
\pmsynonym{volume form}{VolumeElement}
\pmsynonym{volume measure}{VolumeElement}
\pmdefines{area element}
\pmdefines{area form}
\pmdefines{area measure}
\pmdefines{Euclidean volume element}
\pmdefines{Euclidean volume form}
\pmdefines{euclidean volume measure}
\pmdefines{surface area measure}
\pmdefines{surface area element}
\pmdefines{surface area form}

\endmetadata

% this is the default PlanetMath preamble.  as your knowledge
% of TeX increases, you will probably want to edit this, but
% it should be fine as is for beginners.

% almost certainly you want these
\usepackage{amssymb}
\usepackage{amsmath}
\usepackage{amsfonts}

% used for TeXing text within eps files
%\usepackage{psfrag}
% need this for including graphics (\includegraphics)
%\usepackage{graphicx}
% for neatly defining theorems and propositions
\usepackage{amsthm}
% making logically defined graphics
%%%\usepackage{xypic}

% there are many more packages, add them here as you need them

% define commands here
\theoremstyle{theorem}
\newtheorem*{thm}{Theorem}
\newtheorem*{lemma}{Lemma}
\newtheorem*{conj}{Conjecture}
\newtheorem*{cor}{Corollary}
\newtheorem*{example}{Example}
\newtheorem*{prop}{Proposition}
\theoremstyle{definition}
\newtheorem*{defn}{Definition}
\theoremstyle{remark}
\newtheorem*{rmk}{Remark}

\begin{document}
If $M$ is an $n$ dimensional manifold, then a \PMlinkname{differential $n$ form}{DifferentialForms} that is never zero is called a {\em volume element}
or a {\em volume form}.  Usually one volume form is associated with the manifold.  The volume element is sometimes denoted
by $dV,$ $\omega$ or $\operatorname{vol}_n.$
If the manifold is a Riemannian manifold with \PMlinkescapetext{metric} $g,$ then the natural volume form is defined in local coordinates $x^1 \ldots x^n$ by
\begin{equation*}
dV := \sqrt{\lvert g \rvert} dx^1 ~ \wedge \ldots \wedge ~dx^n .
\end{equation*}
It is not hard to show that a manifold has a volume form if and only if it is orientable.

If the manifold is ${\mathbb{R}}^n,$ then
the usual volume element $dV = dx^1~ \wedge ~ dx^2 ~ \wedge \ldots \wedge ~dx^n$ is called the {\em Euclidean volume element}
or {\em Euclidean volume form}.
In this context, ${\mathbb{C}}^n$ is usually treated as ${\mathbb{R}}^{2n}$ unless stated otherwise.

When $n=2$, then the form is frequently called the {\em area element} or {\em area form} and frequently denoted
by $dA$.  Furthermore, when the manifold is a submanifold of ${\mathbb{R}}^3$, then many authors will refer to
a {\em surface area element} or {\em surface area form}.

When the context is measure theoretic, this form is sometimes called a {\em volume measure}, {\em area measure},
etc...

\begin{thebibliography}{9}
\bibitem{spivak}
Michael Spivak.
{\em \PMlinkescapetext{Calculus on Manifolds}},
W.A. Benjamin, Inc., 1965.
\bibitem{boothby}
William M.\@ Boothby.
{\em \PMlinkescapetext{An Introduction to Differentiable Manifolds and
Riemannian Geometry}},
Academic Press, San Diego, California, 2003.
\end{thebibliography}
%%%%%
%%%%%
\end{document}
