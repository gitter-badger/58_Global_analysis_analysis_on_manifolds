\documentclass[12pt]{article}
\usepackage{pmmeta}
\pmcanonicalname{ProofThatTransitionFunctionsOfCotangentBundleAreValid}
\pmcreated{2013-03-22 14:52:25}
\pmmodified{2013-03-22 14:52:25}
\pmowner{rspuzio}{6075}
\pmmodifier{rspuzio}{6075}
\pmtitle{proof that transition functions of cotangent bundle are valid}
\pmrecord{13}{36550}
\pmprivacy{1}
\pmauthor{rspuzio}{6075}
\pmtype{Proof}
\pmcomment{trigger rebuild}
\pmclassification{msc}{58A32}

% this is the default PlanetMath preamble.  as your knowledge
% of TeX increases, you will probably want to edit this, but
% it should be fine as is for beginners.

% almost certainly you want these
\usepackage{amssymb}
\usepackage{amsmath}
\usepackage{amsfonts}

% used for TeXing text within eps files
%\usepackage{psfrag}
% need this for including graphics (\includegraphics)
%\usepackage{graphicx}
% for neatly defining theorems and propositions
%\usepackage{amsthm}
% making logically defined graphics
%%%\usepackage{xypic}

% there are many more packages, add them here as you need them

% define commands here
\begin{document}
In this entry, we shall verify that the transition functions proposed for the cotangent bundle \PMlinkescapeword{satisfy} the three criteria required by the classical definition of a manifold.

The first criterion is the easiest to verify.  If $\alpha = \beta$, then $\sigma_{\alpha \alpha}$ reduces to the identity and we have
 $$\bigg({\sigma'}_{\alpha \alpha} (x_1, \ldots, x_{2n}) \bigg)^i = \bigg(\sigma_{\alpha \alpha} (x_1, \ldots, x_n) \bigg)^i = x^i \qquad 1 \le i \le n $$
$$\bigg({\sigma'}_{\alpha \alpha} (x_1, \ldots, x_{2n}) \bigg)^{i+n} = \sum_{j = 1}^n {\partial \bigg(\sigma_{\alpha \alpha} (x_1, \ldots, x_n) \bigg)^i \over \partial x_j} x^{j+n} = \sum_{j = 1}^n {\partial x^i \over \partial x_j} x^{j+n} = x^{i+n} \qquad 1 \le i \le n$$
Thus we see that ${\sigma'}_{\alpha \alpha}$ is the identity map, as required.

Next, we turn our attention to the third criterion --- showing that ${\sigma'}_{\beta \gamma} \circ {\sigma'}_{\alpha \beta} = {\sigma'}_{\alpha \gamma}$ .  For clarity of notation let us define $y^i = ({\sigma'}_{\alpha \beta})^i (x^1, \ldots x^{2n})$.  Then we have
\begin{eqnarray*}
({\sigma'}_{\beta \gamma} \circ {\sigma'}_{\alpha \beta})^i (x^1, \dots, x^{2n}) &=& ({\sigma'}_{\beta \gamma})^i (y^1, \dots, y^{2n}) \\
&=& (\sigma_{\beta \gamma})^i (y^1, \dots, y^n) \\
&=& (\sigma_{\beta \gamma} \circ \sigma_{\alpha \beta})^i (x^1, \dots, x^n) \\
&=& (\sigma_{\alpha \gamma})^i (x^1, \dots, x^n) \\
&=& ({\sigma'}_{\alpha \gamma})^i (x^1, \dots, x^{2n}) \\
\end{eqnarray*}
when $1 \le i \le n $.
\begin{eqnarray*}
({\sigma'}_{\beta \gamma} \circ {\sigma'}_{\alpha \beta})^{i+n} (x^1, \dots, x^{2n}) &=& ({\sigma'}_{\beta \gamma})^{i+n} (y^1, \dots, y^{2n}) \\
&=& \sum_{j = 1}^n {\partial \bigg(\sigma_{\beta \gamma} (y_1, \ldots, y_n) \bigg)^i \over \partial y_j} y^{j+n} \\
&=& \sum_{j = 1}^n \sum_{k = 1}^n {\partial \bigg(\sigma_{\beta \gamma} (y_1, \ldots, y_n) \bigg)^i \over \partial y_j} {\partial \bigg(\sigma_{\alpha \beta} (x_1, \ldots, x_n) \bigg)^j \over \partial x_k} x^{n+k} \\
&=& \sum_{k = 1}^n {\partial \bigg(\sigma_{\beta \gamma} \circ \sigma_{\alpha \beta} (x_1, \ldots, x_n) \bigg)^i \over \partial x_k} x^{n+k} \\
&=& \sum_{k = 1}^n {\partial \bigg(\sigma_{\alpha \gamma} (x_1, \ldots, x_n) \bigg)^i \over \partial x_k} x^{n+k} \\
&=& {\sigma'}_{\alpha \gamma} (x^1, \dots, x^{2n}) \\
\end{eqnarray*}
when $1 \le i \le n $.

Finally, the second criterion does not need to be checked because it is a consequence of the first and third criteria.
%%%%%
%%%%%
\end{document}
