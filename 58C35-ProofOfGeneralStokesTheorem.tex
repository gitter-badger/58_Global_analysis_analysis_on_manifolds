\documentclass[12pt]{article}
\usepackage{pmmeta}
\pmcanonicalname{ProofOfGeneralStokesTheorem}
\pmcreated{2013-03-22 13:41:43}
\pmmodified{2013-03-22 13:41:43}
\pmowner{paolini}{1187}
\pmmodifier{paolini}{1187}
\pmtitle{proof of general Stokes theorem}
\pmrecord{9}{34370}
\pmprivacy{1}
\pmauthor{paolini}{1187}
\pmtype{Proof}
\pmcomment{trigger rebuild}
\pmclassification{msc}{58C35}

% this is the default PlanetMath preamble.  as your knowledge
% of TeX increases, you will probably want to edit this, but
% it should be fine as is for beginners.

% almost certainly you want these
\usepackage{amssymb}
\usepackage{amsmath}
\usepackage{amsfonts}

% used for TeXing text within eps files
%\usepackage{psfrag}
% need this for including graphics (\includegraphics)
%\usepackage{graphicx}
% for neatly defining theorems and propositions
%\usepackage{amsthm}
% making logically defined graphics
%%%\usepackage{xypic}

% there are many more packages, add them here as you need them

% define commands here
\begin{document}
We divide the proof in several steps.

\emph{Step One.}

Suppose $M=(0,1]\times (0,1)^{n-1}$ and 
\[
        \omega(x_1,\ldots,x_n)=f(x_1,\ldots,x_n)\, d
        x_1\wedge\cdots\wedge \widehat{d x_j}\wedge \cdots \wedge d
        x_n
\]
(i.e. the term $dx_j$ is missing).
Hence we have
\begin{eqnarray*}
        d\omega(x_1,\ldots,x_n)
        &=&
        \left(\frac{\partial f}{\partial x_1} d x_1+\cdots +\frac{\partial
        f}{\partial x_n}d x_n\right)\wedge d x_1\wedge \cdots \wedge
        \widehat{d x_j}\wedge \cdots \wedge d x_n \\
        &=& (-1)^{j-1} \frac{\partial f}{\partial x_j} d x_1\wedge
        \cdots \wedge d x_n
\end{eqnarray*}
and from the definition of integral on a manifold we get
\[
        \int_M d\omega
        = \int_0^1\cdots \int_0^1 (-1)^{j-1} \frac{\partial
f}{\partial x_j} d x_1 \cdots d x_n.
\]
From the fundamental theorem of Calculus we get
\[
        \int_M d\omega
        =(-1)^{j-1}
        \int_0^1\cdots\widehat{\int_0^1}\cdots\int_0^1
        f(x_1,\ldots,1,\ldots,x_n)-f(x_1,\ldots,0,\ldots,x_n) d
        x_1\cdots \widehat{ d x_j}\cdots d x_n.
\]
Since $\omega$ and hence  $f$ have compact support in $M$ we obtain
\[
        \int_M d\omega=\left\{
        \begin{array}{lcl}
                \int_0^1\cdots\int_0^1 f(1,x_2,\ldots,x_n) d x_2
        \cdots d x_n & \text{if} & j=1 \\\\
                0 &\text{if} & j>1 .
        \end{array}\right.
\]

On the other hand we notice that
 $\int_{\partial M}\omega$
is to be understood as $\int_{\partial M}i^* \omega$ where
$i:\partial M\to M$ is the inclusion map.
Hence it is trivial to verify that when $j\neq 1$ then $i^*\omega=0$ 
while if $j=1$ it holds
\[
        i^*\omega(x) = f(1,x_2,\ldots,x_n) d x_2\wedge\ldots\wedge d x_n
\]
and hence, as wanted
\[
        \int_{\partial M}i^*\omega
        = \int_0^1\cdots\int_0^1 f(1,x_2,\ldots,x_n) d x_2 \cdots d x_n.
\]

\emph{Step Two.}

Suppose now that $M=(0,1]\times (0,1)^{n-1}$ and let $\omega$ be any differential form.
We can always write
        \[
        \omega(x)=\sum_j f_j(x) d x_1 \wedge\cdots \wedge \widehat{
d x_j} \wedge \cdots \wedge d x_n
        \]
and by the additivity of the integral we can reduce ourself to the previous case.

\emph{Step Three.}

When $M=(0,1)^n$ we could follow the proof as in the first case
and end up with
$\int_M d \omega = 0$ 
while, in fact, $\partial
M=\emptyset$.

\emph{Step Four.}

Consider now the general case.

First of all we consider an oriented atlas $(U_i,\phi_i)$ such that either 
$U_i$ is the cube
$(0,1]\times (0,1)^{n-1}$ or $U_i=(0,1)^n$. 
This is always possible. In fact given any open set $U$ in $[0,+\infty)\times
\mathbb R^{n-1}$ and a point $x\in U$ up to translations and rescaling it is possible
to find a ``cubic'' neighbourhood of $x$ contained in $U$.

Then consider a partition of unity $\alpha_i$ for this atlas.

From the properties of the integral on manifolds we have
\begin{eqnarray*}
        \int_{M} d \omega
        &=& \sum_i \int_{U_i} \alpha_i \phi^* d \omega
        = \sum_i \int_{U_i} \alpha_i d (\phi^* \omega) \\
        &=& \sum_i \int_{U_i} d (\alpha_i\cdot \phi^*\omega)
                - \sum_i \int_{U_i} (d\alpha_i)\wedge(\phi^*\omega).
\end{eqnarray*}

The second integral in the last equality is zero since
$\sum_i
d\alpha_i= d \sum_i \alpha_i = 0$, while applying the previous steps to the first integral we have
\[
        \int_{M} d\omega = \sum_i \int_{\partial U_i}\alpha_i\cdot
        \phi^*\omega.
\]
On the other hand, being
$(\partial U_i,\phi_{|\partial U_i})$ an oriented atlas for $\partial M$ and being ${\alpha_i}_{|\partial U_i}$ a partition of unity, we have
\[
        \int_{\partial M} \omega
        = \sum_i \int_{\partial U_i} \alpha_i \phi^* \omega
\]
and the theorem is proved.
%%%%%
%%%%%
\end{document}
