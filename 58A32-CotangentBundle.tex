\documentclass[12pt]{article}
\usepackage{pmmeta}
\pmcanonicalname{CotangentBundle}
\pmcreated{2013-03-22 13:59:02}
\pmmodified{2013-03-22 13:59:02}
\pmowner{rspuzio}{6075}
\pmmodifier{rspuzio}{6075}
\pmtitle{cotangent bundle}
\pmrecord{17}{34757}
\pmprivacy{1}
\pmauthor{rspuzio}{6075}
\pmtype{Definition}
\pmcomment{trigger rebuild}
\pmclassification{msc}{58A32}

% this is the default PlanetMath preamble.  as your knowledge
% of TeX increases, you will probably want to edit this, but
% it should be fine as is for beginners.

% almost certainly you want these
\usepackage{amssymb}
\usepackage{amsmath}
\usepackage{amsfonts}

% used for TeXing text within eps files
%\usepackage{psfrag}
% need this for including graphics (\includegraphics)
%\usepackage{graphicx}
% for neatly defining theorems and propositions
%\usepackage{amsthm}
% making logically defined graphics
%%%\usepackage{xypic}

% there are many more packages, add them here as you need them

% define commands here
\newtheorem{thm}{Theorem}
\newtheorem{prop}{Proposition}

\newcommand{\ab}[1]{{#1}_{\mathrm{ab}}}
\newcommand{\Ad}{\mathrm{Ad}}
\newcommand{\ad}{\mathrm{ad}}
\newcommand{\Aut}{\mathrm{Aut}\,}
\newcommand{\Aff}[2]{\mathrm{Aff}_{#1} #2}
\newcommand{\aff}[2]{\mathfrak{aff}_{#1} #2}
\newcommand{\mcB}{\mathcal{B}}
\newcommand{\bb}[1]{\mathbb{#1}}
\newcommand{\bfrac}[2]{\left[\frac{#1}{#2}\\right]}
\newcommand{\bkh}{\backslash}
\newcommand{\Cyc}[2]{\mathcal{C}^{#1}_{#2}}
\newcommand{\Cbar}[2]{\overline{\C{#1}{#2}}}
%\newcommand{\CD}{\R[\Delta]}
\newcommand{\C}{\mathbb{C}}
\newcommand{\CF}[2]{\ensuremath{\mathfrak{C}(#1,#2)}}
\newcommand{\Cinf}{\EuScript{C}^{\infty}}
\newcommand{\cmp}{cyclic mod $p$\xspace}
\newcommand{\cp}{\mathrm{c.p.}}
\newcommand{\CS}{\EuScript{CS}}
\newcommand{\deck}{\EuScript{D}}
\newcommand{\defl}[1]{\mathfrak{def}_{#1}}
\newcommand{\Der}{\mathrm{Der}\,}
\newcommand{\eH}{[X_H]-[Y_H]}
\newcommand{\EL}{\mathcal{EL}}
\newcommand{\End}{\mathrm{End}}
\newcommand{\ES}[1]{\EuScript{#1}}
\newcommand{\Ext}{\mathrm{Ext}}
\newcommand{\Fix}{\mathrm{Fix}}
\newcommand{\fr}[1]{\mathfrak{#1}}
\newcommand{\Frat}{\mathrm{Frat}\,}
\newcommand{\Gal}[1]{\Gamma(#1 |\Q)}
\newcommand{\GL}[2]{\mathrm{GL}_{#1} #2}
\newcommand{\gl}[2]{\mathfrak{gl}_{#1} #2}
\newcommand{\GrR}[1]{a(#1 G)}
\newcommand{\Gr}{\mathrm{Gr}\,}
\newcommand{\mcH}{\mathcal{H}}
\renewcommand{\H}{\mathbb{H}}
\newcommand{\Hom}[2]{\mathrm{Hom}(#1,#2)}
\newcommand{\id}{\mathrm{id}}
\newcommand{\im}{\mathrm{im}}
\newcommand{\ind}[2]{\mathrm{ind}^{#1}_{#2}}
\newcommand{\indp}[2]{\mathfrak{ind}^{#1}_{#2}}
\renewcommand{\inf}[1]{\mathfrak{inf}_{#1}}
\newcommand{\inn}[1]{\langle #1\rangle}
\renewcommand{\int}{\mathrm{int}}
\newcommand{\Iso}{\mathrm{Iso}}
\newcommand{\K}{\mathcal{K}}
\renewcommand{\ker}{\mathrm{ker}\,}
\renewcommand{\L}[1]{\mathfrak{L}(#1)}
\newcommand{\lap}[1]{\Delta_{#1}}
\newcommand{\lapM}{\Delta_M}
\newcommand{\Lie}{\mathrm{Lie}}
\newcommand{\lineq}{linearly equivalent\xspace}
\newcommand{\mc}[1]{\mathcal{#1}}
\newcommand{\mG}{m_G}
\newcommand{\mK}{m_{\K}}
\newcommand{\mindeg}[1]{\fr{md}(#1)}
\newcommand{\N}{\mathbb{N}}
\renewcommand{\O}{\mathcal{O}}
\newcommand{\Om}{\Omega}
\newcommand{\om}{\omega}
\newcommand{\Orb}{\mathrm{Orb}}
\newcommand{\pad}{\hat{\Z}_p}
\newcommand{\pder}[2]{\frac{\partial #1}{\partial #2}}
\newcommand{\pderw}[1]{\frac{\partial}{\partial #1}}
\newcommand{\pdersec}[2]{\frac{\partial^2 #1}{\partial {#2}^2}} 
\newcommand{\perm}[1]{\pi_{#1}}
\newcommand{\Q}{\mathbb{Q}}
\newcommand{\R}{\mathbb{R}}
\newcommand{\rad}{\mathrm{rad}\,}
\newcommand{\res}[2]{\mathrm{res}^{#1}_{#2}}
\newcommand{\resp}[2]{\mathfrak{res}^{#1}_{#2}}
\newcommand{\RG}{\EuScript{R}_G}
\newcommand{\rk}{\mathrm{rk}\,}
\newcommand{\V}[1]{\mathbf{#1}}
\newcommand{\vp}{\varphi}
\newcommand{\Stab}{\mathrm{Stab}}
\newcommand{\SL}[2]{\mathrm{SL}_{#1} #2}
\renewcommand{\sl}[2]{\fr{sl}_{#1} #2}
\newcommand{\SO}[2]{\mathrm{SO}_{#1} #2}
\newcommand{\Sp}[2]{\mathrm{Sp}_{#1} #2}
\renewcommand{\sp}[2]{\fr{sp}_{#1} #2}
\newcommand{\SU}[1]{\mathrm{SU}( #1)}
\newcommand{\su}[1]{\fr{su}_{#1}}
\newcommand{\Sym}{\mathrm{Sym}}
\newcommand{\sym}{\mathrm{sym}}
\newcommand{\Tg}{\mc{T}(\fr g)}
\newcommand{\tom}{\tilde{\omega}}
\newcommand{\ghtghp}{\fr g/\fr h\oplus(\fr g/\fr h^\perp)^*}
\newcommand{\ghps}{(\fr g/\fr h^\perp)^*}
\newcommand{\Tr}{\mathrm{Tr}}
\newcommand{\tr}{\mathrm{tr}}
%\renewcommand{\thechapter}{\Roman{chapter}}
%\renewcommand{\thesection}{\thechapter.\arabic{section}}
%\renewcommand{\thethm}{\thechapter.\arabic{thm}}
\newcommand{\Ug}{\mc{U}(\fr g)}
\newcommand{\Uh}{\mc{U}(\fr h)}
\renewcommand{\V}[1]{\mathbf{#1}}
\newcommand{\Z}{\mathbb{Z}}
\newcommand{\Zp}{\Z/p}
\begin{document}
{\bf Overview}

Let $M$ be a differentiable manifold.  Analogously to the construction of the tangent bundle, we can make the set of covectors on a given manifold into a vector bundle over $M$, denoted $T^*M$ and called the {\em cotangent} bundle of $M$.  

{\bf Rigorous Definition}

To make this definition precise it is convenient to use the \PMlinkname{classical definition of a manifold}{NotesOnTheClassicalDefinitionOfAManifold}.  Let $M$ be an $n$-dimensional differentiable manifold, let $\{V_\alpha \mid \alpha \in {\cal A}\}$ (each $V_\alpha$ is an open subset of $\mathbb{R}^n$) be an atlas of $M$ with transition functions $\sigma_{\alpha \beta}$.

As an atlas for $T^* (M)$, we may take $\{V_\alpha \times \mathbb{R}^n \mid \alpha \in {\cal A}\}$.  We may construct transition functions ${\sigma'}_{\alpha \beta}$ as follows:
 $$\bigg({\sigma'}_{\alpha \beta} (x^1, \ldots, x^{2n}) \bigg)^i = \bigg(\sigma_{\alpha \beta} (x^1, \ldots, x^n) \bigg)^i \qquad 1 \le i \le n$$
 $$\bigg({\sigma'}_{\alpha \beta} (x^1, \ldots, x^{2n}) \bigg)^{i+n} = \sum_{j = 1}^n {\partial \bigg(\sigma_{\alpha \beta} (x^1, \ldots, x^n) \bigg)^i \over \partial x^j} x^{j+n} \qquad 1 \le i \le n$$
For these to be valid transition functions, they must satisfy the three criteria.  For a verification that these criteria are satisfied, please see the attachment.

{\bf Bundle Structure}

The cotangent bundle is a $GL(n)$ vector bundle over the manifold $M$.  To substantiate this claim, we must specify a projection map onto the manifold $M$ and local trivializations and transition functions and verify that they satisfies the defining properties of a bundle.  In terms of the local coordinates used above, it is easy to describe the projection map $\pi$:
 $${\pi (x^1, \ldots, x^{2n})}^i = x^i$$
The local trivializations are also somewhat trivial:
 $${\phi_\alpha (x^1, \ldots, x^{2n})} = x^{i+n}$$
Finally, the transition functions are given as follows:
 $$g_{\alpha \beta} (x^1, \ldots, x^{2n})^i_j = {\partial \big( \sigma_{\alpha \beta} (x^1, \ldots x^n) \big)^i \over \partial x^j}$$

For a verification that $( T^* M, \pi, \phi_\alpha, g_{\alpha \beta} )$ satisfies the three criteria for a bundle, please see the attachment.

{\bf Properties}

The cotangent bundle $T^*M$ is the vector bundle dual to the tangent bundle $TM$.  On any differentiable manifold, $T^*M \cong TM$ (for example, by the existence of a Riemannian metric), but this identification is by no means canonical, and thus it is useful to distinguish between these two objects. 

 The cotangent bundle to any manifold has a natural symplectic structure given in terms of the Poincar\'e 1-form, which is in some sense unique.  This is not true of the tangent bundle.  The existence of a symplectic structure implies that the cotangent bundle is always orientable, even if the original manifold is not.
%%%%%
%%%%%
\end{document}
